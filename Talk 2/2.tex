\section{Formal type theory}

Since we need to interpret the rules and axioms of type theory, we need a more careful treatment of the formal system of type theory.

By a context, we mean a finite (possibly empty) list of distinct variables with their respective typings:
\[x_1:A_1,...,x_n:A_n\]

Contexts are often denoted by $\Gamma$, $\Delta$, etc.

We have five forms of judgments:
\begin{enumerate}
    \item $\Gamma \textsf{ ctx}$
    \item $\Gamma \vdash A \textsf{ type}$
    \item $\Gamma \vdash a : A$
    \item $\Gamma \vdash A \equiv B  \textsf{ type}$
    \item $\Gamma \vdash a \equiv b : A$
\end{enumerate}

The derivibility of those judements are defined inductively by the \textit{inference rules} of type theory.

Let $B$ be a term / type / context, write $B[a/x]$ for the \textit{substitution} of a term $a$ for free occurrences of the variable $x$ in the term $B$. Moreover, let $B[a_1,...,a_n/x_1,...,x_n]$ denote the simultaneous substitution.
