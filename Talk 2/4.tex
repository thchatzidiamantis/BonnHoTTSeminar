\section{Problems of the naîve model}

\subsection{Coherence}
Type-theoretically substitution is a strictly functorial operation, whereas its categorical counterpart, pullback, without making any choices, is functorial only up to isomorphism.

Consider $$\vdash a :A \quad x: A\vdash b:B \quad x:A,y:B\vdash T \type.$$ Then we can do substitution twice and obtain $$\vdash (T[b/y])[a/x] \type.$$ This process has semantics as performing pullback twice:

\[\begin{tikzcd}
	{(T[b/y])[a/x]} & {T[b/y]} & T \\
	1 & A & B
	\arrow[from=1-1, to=1-2]
	\arrow[dashed, from=1-1, to=2-1]
	\arrow[from=1-2, to=1-3]
	\arrow[from=1-2, to=2-2]
	\arrow[from=1-3, to=2-3]
	\arrow[""{name=0, anchor=center, inner sep=0}, "{a}", from=2-1, to=2-2]
	\arrow[""{name=1, anchor=center, inner sep=0}, "{b}", from=2-2, to=2-3]
	\arrow["\lrcorner"{anchor=center, pos=0.125}, draw=none, from=1-1, to=0]
	\arrow["\lrcorner"{anchor=center, pos=0.125}, draw=none, from=1-2, to=1]
\end{tikzcd}\]

On the other hand, we have, syntatically, $$(T[b/y])[a/x] = (T[a/x])[(b[a/x])/y].$$
And the right formula has semantics:

\[\begin{tikzcd}
	{(T[a/x])[(b[a/x])/y]} & {T[a/x]} & T \\
	1 & {B[a/x]} & B \\
	& 1 & A
	\arrow[from=1-1, to=1-2]
	\arrow[dashed, from=1-1, to=2-1]
	\arrow[from=1-2, to=1-3]
	\arrow[from=1-2, to=2-2]
	\arrow[from=1-3, to=2-3]
	\arrow[""{name=0, anchor=center, inner sep=0}, "{b[a/x]}", from=2-1, to=2-2]
	\arrow[""{name=1, anchor=center, inner sep=0}, from=2-2, to=2-3]
	\arrow[from=2-2, to=3-2]
	\arrow[from=2-3, to=3-3]
	\arrow[""{name=2, anchor=center, inner sep=0}, "{a}", from=3-2, to=3-3]
	\arrow["\lrcorner"{anchor=center, pos=0.125}, draw=none, from=1-1, to=0]
	\arrow["\lrcorner"{anchor=center, pos=0.125}, draw=none, from=1-2, to=1]
	\arrow["\lrcorner"{anchor=center, pos=0.125}, draw=none, from=2-2, to=2]
\end{tikzcd}\]

One can show that the resulting maps are isomorphic but not neccessarily equal if the pullbacks are chosen arbitrarily.

One possible solution is to make pullbacks a definite data of a category. (For example, a category with attributes.) In this way, making substitution is no longer arbitrary but under control.

\subsection{Intensional vs. extensional}
A type theory with $\Id$-type is said to be extensional if the \textit{reflection} rule is derivable:
\begin{center}
    \AxiomC{%$\Gamma\vdash a : A$ \quad $\Gamma\vdash b: A$ \quad 
    $\Gamma \vdash p : \Id_A(a,b)$}
    \RightLabel{$\Id$-\scriptsize{REFL}}
    \UnaryInfC{$\Gamma\vdash a \equiv b : A$}
    \DisplayProof
\end{center}

The interpretation above does validate $\Id$-{\scriptsize{REFL}} because 
%is sound, then the type theory is extensional, simply because 
%the following equalizer admits a section iff $a=b$ as morphisms:
if the following equalizer admits a section, then $a=b$.
$$\begin{tikzcd}
    {\Id_A(a,b)} & {\Gamma} & {\Gamma.A}
    \arrow["{\text{equalizer}}", from=1-1, to=1-2]
    \arrow["b"', shift right, from=1-2, to=1-3]
    \arrow["a", shift left, from=1-2, to=1-3]
\end{tikzcd}$$
Hence $\Id$-{\scriptsize{REFL}} is indeed consistent with MLTT, but is it derivable? 
%A crucial deficiency of extensional type theory is that type checking is not decidable. Then is this rule derivable in MLTT? 
The question is answered negetively by a counter model given Hofmann and Streicher \cite{hofmann1998groupoid}.

\subsection{Universe and univalence}
Don't forget the axiom making the HoTT distinguished. It's a fact that it is consistent with MLTT \cite{kapulkin2021simplicial} but constructing a model is way more complicate and will be left for my further talks.