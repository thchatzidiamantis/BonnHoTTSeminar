\documentclass{article}

\usepackage{amsmath,amsthm,latexsym,amssymb}
\usepackage{mathtools}
\usepackage{amssymb}
\usepackage{bbm}
\usepackage[mathscr]{euscript}
\usepackage{libertine}
\usepackage{mathrsfs}
\usepackage[mathscr]{euscript}


\usepackage[all,arc]{xy}
\usepackage{tikz}
\usepackage{tikz-cd}

\usepackage[hidelinks]{hyperref}
\hypersetup{colorlinks=true, linkcolor=teal, urlcolor=teal, citecolor=purple}

\newcommand{\textbi}[1]{\textbf{\textit{#1}}}

\newcommand{\C}{\mathscr{C}}
\newcommand{\D}{\mathscr{D}}
\newcommand{\F}{\mathscr{F}}
\newcommand{\N}{\mathscr{N}}
\newcommand{\bN}{\mathbb{N}}
\newcommand{\bZ}{\mathbb{Z}}
\newcommand{\bR}{\mathbb{R}}
\newcommand{\cS}{\mathcal{S}}
\newcommand{\X}{\mathscr{X}}

\newcommand{\Top}{\mathscr{T}\mathrm{op}}
\newcommand{\id}{\mathrm{id}}
\newcommand{\Obj}{\mathrm{Obj}}
\newcommand{\Ho}{\mathrm{Ho}}
\newcommand{\Hom}{\mathrm{Hom}}
\newcommand{\const}{\mathrm{const}}
\newcommand{\Set}{\cS\mathrm{et}}
\newcommand{\sS}{s\cS}
\newcommand{\Fun}{\mathrm{Fun}}
\newcommand{\Aut}{\mathrm{Aut}}
\newcommand{\Map}{\mathrm{Map}}
\newcommand{\Nat}{\mathrm{Nat}}
\newcommand{\Sq}{\mathrm{Sq}}
\newcommand{\colim}{\mathrm{colim}}

\newcommand{\tProp}{\mathsf{Prop}}
\newcommand{\tSet}{\mathsf{Set}}
\newcommand{\tType}{\mathsf{Type}}


\newcommand{\chapterauthor}[1]{%
  {\parindent0pt\vspace*{-10pt}%
  \linespread{1.1}\fontsize{16}{12}\selectfont#1%
  \par\nobreak\vspace*{30pt}}
}


\swapnumbers
\newtheorem{prop}[subsection]{Proposition}
\newtheorem{cor}[subsection]{Corollary}
\newtheorem{lem}[subsection]{Lemma}
\newtheorem{thm}[subsection]{Theorem}

\theoremstyle{definition}
\newtheorem{nota}[subsection]{Notation}
\newtheorem{defin}[subsection]{Definition}
\newtheorem{rem}[subsection]{Remark}
\newtheorem{ex}[subsection]{Example}
\newtheorem{cons}[subsection]{Construction}


\usepackage[textwidth=2.65cm,textsize=scriptsize,color=orange!40,linecolor=green!40!black,colorinlistoftodos]{todonotes} 

\title{Higher Categories for the HoTT-Headed Mathematician}
\author{Theofanis Chatzidiamantis-Christoforidis}
\date{June 2024}

\begin{document}
\maketitle
\tableofcontents

\begin{abstract}
    This is the second in a series of talks with the end goal of introducing simplicial type theory. In this talk, our aim is to provide a (non-precise) introduction to Kan complexes, quasicategories, and complete Segal spaces, and discuss the limitations of ``regular" HoTT when it comes to working with higher categories that are not Kan complexes.
\end{abstract}

\section{Motivation(s)}

We will investigate a common thread between different areas of mathematics, all pointing towards a need to codify some sort of higher structure.

\subsection*{Categories}

We start with an example from category theory itself: It is often useful to define, up to set-theoretic size issues (which one can fix by using large enough cardinals), a \textit{category of categories} $\mathscr{C}\mathrm{at}$, where objects are categories and morphisms are functors. Then, standard category theory definitions should tell us that equivalences in this category are precisely functors that have strict inverses, i.e., isomorphisms of categories. This is, however, not the notion of equivalence we usually want in category theory: equivalences of categories are just fine most of the time!
\par To describe equivalences of categories, we go to one more fundamental type of ``map'' in category theory: \textit{natural transformations}. Then, an equivalence of categories can be defined as a pair of functors 
$$F:\C\to \D,\quad G:\D\to \C$$
such that there are natural isomorphisms $$GF\xRightarrow{\cong}id_\C, \quad FG\xRightarrow{\cong}id_\D$$
To get around this issue, we say that $\mathscr{C}\mathrm{at}$ is a \textbi{2-category}, i.e., a category that has objects and morphisms but also has \textit{$2$-morphisms} which are morphisms between morphisms, and with the notion of equivalence between two objects appropriately weakened. 
\\ \\
What we will see in the next example is that this is not the only thing it makes sense to weaken: Associativity and uniqueness conditions also do not need to be satisfied strictly to give us a sensible generalization of ordinary category theory. In the language of $2$-categories, we have the notion of a \textbi{bicategory} or \textit{weak $2$-category}, which is equipped with natural isomorphisms witnessing associativity and the properties of identity morphisms. \par
 So, there is now at least a reason to want to go even higher: One may want to weaken these conditions as well, having witnesses of higher coherence conditions. An $\infty$-category should be a way to package all these conditions in a single, reasonably short, definition.

\subsection*{Homotopy theory}

Consider your favourite category of topological spaces. In homotopy theory, what we want is to try and study spaces up to homotopy equivalence, and, in some cases, up to weak equivalence (for nice enough spaces, these coincide). Applying a category-theoretic framework to this is quite problematic: Take, for instance, definitions using universal properties. These are designed to provide \textit{unique characterizations up to isomorphism}, which, in topological spaces, is a much stronger notion than homotopy equivalence. 

\begin{ex}
    Limits in topological spaces are not homotopy-invariant:
    In the diagram
    \[
    \begin{tikzcd}[sep=small]
        & * \arrow[rr] \arrow[dd] & & * \arrow[from = dl, "\sim"] \arrow[dd] \\
        \varnothing \arrow[ur, dotted] \arrow[rr, crossing over] \arrow[dd] & & I  & \\ 
        & * \arrow[rr] \arrow[from = dl, "\sim"] & & * \arrow[from = dl, "\sim"] \\
        \{2\} \arrow[rr] & & \bR \arrow[from = uu, crossing over] &
    \end{tikzcd}
    \]
    the front and the back squares are pullbacks, but $\varnothing\not\simeq *$.
\end{ex}

One might think that to fix this, all we have to do is somehow ``mod out homotopy'', but this does not work: There are higher coherences!
\par Between any two spaces we have maps and homotopies between maps, but we can also have homotopies between homotopies, and homotopies between those. Once again, what we really need to replace our category of spaces with is something with higher and higher levels of morphisms, exhibithing all coherence conditions up to homotopy. Moreover, note that homotopy is an equivalence relation: we can think of all homotopies as being invertible. In category-theoretic language, we will have all our higher morphisms be equivalences. We will call such higher categories $(\infty,1)$-\textbi{categories}.\\ \\ 
We can use the same ideas to describe the internal structure of a topological space. Pick a space $X$ and let $I$ be the unit interval. Then maps $I\to X$ are paths in $X$, and our description can now be applied to the internal structure of $X$ as well. Similarly mapping out of spheres and fixing a basepoint, we get homotopy groups, which, once again, have infinite levels.
\par In higher category theory, we will represent this by a form of $\infty$-category where all morphisms are equivalences, which, analogously to $1$-category theory, we call an $\infty$-\textbi{groupoid}.


\subsection*{Type theory}

Recall that in Martin-L\"of type theory, we have intensional identity types that may contain non-trivial structure. Topologically, we interpreted the identity type $a=_A b$ as paths between $a$ and $b$. Then, since the $a=_A b$ is itself a type, we can just go higher and consider identifications of paths $p,q:a=_A b$ in the identity type, getting types of the form $$p=_{a=_A b}q$$ These are our homotopies! Once again, we can have identifications between identifications of identifications, etc. 
\par We also have associativity of paths, so we have a true higher groupoidal structure.
\begin{rem}
    We can use the same structure to define homotopy groups of types: For a type $A$ and a chosen basepoint $a:A$, we can define the loop space $$\Omega(A,a)\coloneqq a=_Aa$$ together with the chosen basepoint $\mathsf{refl}_a$, and iteratively define higher loop spaces $\Omega^{n+1}(A,a)\coloneqq \Omega(\Omega^n(A,a))$. We can then use the set-truncation operation to define the homotopy groups $$\pi_n(A,a)\coloneqq ||\Omega^n(A,a)||_0$$
    This is the starting point of synthetic homotopy theory, recovering many classical results from algebraic topology as type-theoretic statements.
\end{rem}

\section{A Quick Overview of Simplicial Objects}

\begin{defin} \
    \begin{enumerate}
        \item We denote by $\Delta$ the \textbi{simplex category} with objects $[n]\coloneqq \{0,\dots,n\}$ for all $n\in\bN$ and morphisms all nondecreasing maps.
        \item Let $\C$ be a category. The category of \textbi{simplicial objects} in $\C$ is the functor category $\Fun(\Delta^{op}, \C)$.
    \end{enumerate}
\end{defin}

\begin{defin}
    In $\Delta$, we define the \textbi{coface} $\delta_n^i: [n-1]\to [n]$ to be the unique strictly monotone map that skips $i$, and the \textbi{codegeneracy} $\sigma_n^j: [n+1]\to [n]$ to be the map that repeats $j$ in positions $j$ and $j+1$.   
\end{defin}

We can form their induced maps in any simplicial object $X$: \newline We get \textbi{face maps} $$d^n_i\coloneqq (\delta_n^i)^*: X_n\to X_{n-1}$$ and \textbi{degeneracy maps} $$s^n_j\coloneqq (\sigma_n^j)^*: X_n\to X_{n+1}$$ 
We usually skip the $n$ for convenience whenever it is implied.

We will mainly be concerned with two special cases: The category $\cS\coloneqq\Fun(\Delta^{op}, \Set)$  of \textbi{simplicial sets} and the category of \textbi{simplicial spaces} $$\sS\coloneqq\Fun(\Delta^{op}, \cS)$$ We identify $\sS$ with the category of \textbi{bisimplicial sets}, i.e., the functor category $$\Fun(\Delta^{op}\times\Delta^{op}, \Set)$$
\par For a simplicial space $X$ and $n\in \bN$, the simplicial set $X_n$ itself consists of sets $X_{nm}\coloneqq X_n(m)$.

\begin{rem}
    One nice property of functor categories $\Fun(\D,\C)$ is that (co)limits exist and can be computed pointwise if they exist in $\C$. Thus $\cS$ and $\sS$ are (co)complete.
\end{rem}

We now introduce some notation for the representable functors in $\cS$.

\begin{defin}
    Let $n\in\bN$. The \textbi{standard $n$-simplex} $\Delta^n$ is the functor $\Hom(-,[n])$.
\end{defin}

Then, an immediate consequence of the Yoneda lemma is that there is a natural bijection $$\Hom_{\cS}(\Delta^n, X)\cong X_n$$ for every simplicial set X. We call a map $\Delta^n\to X$ (or, equivalently, an element of $X_n$) an $n$-\textit{simplex} of $X$. 

\begin{ex}
    Note that there is only one map $[n]\to [0]$ for every $n$, thus $\Delta^0$ is a final object of $\cS$. Knowing this, it makes sense to think of the $0$-simplices of a simplicial sets as its points.
\end{ex}

To what extent can we repeat this for simplicial spaces? There are two ways that we can think of a simplicial set $X$ as a simplicial space: 
\begin{itemize}
    \item The \textbi{vertical embedding} $i_F(X)_{nm}\coloneqq X_n$, and
    \item The \textbi{horizontal embedding} $i_\Delta(X)_{nm}\coloneqq X_m$
\end{itemize}

\begin{defin}
    For $n\in\bN$ we define $F(n)\coloneqq i_F(\Delta^n)$ and $\Delta[n]\coloneqq  i_\Delta(\Delta^n)$.
\end{defin}

We now recall an important fact about presheaf categories.
\begin{prop}
    Let $\D$ be a small category. Then the category $\Fun(\D^{op}, \Set)$ is cartesian closed, with exponential given by the formula \begin{align}\label{expo}
        B^A(x)=\Nat(\Hom_{\D}(-,x)\times A, B)
    \end{align} for any functors $A,B:\D^{op}\to \Set$ and $x\in\Obj(\D)$.
\end{prop}

Let us first apply this to $\cS$: \ref{expo} tells us precisely that $$(Y^X)_n=\Hom_{\cS}(\Delta^n\times X, Y)$$
For $\sS$, things get a little more complicated: If $Z$ and $W$ are simplicial spaces, we first have to look at them as bisimplicial sets to have a category where \ref{expo} applies. The exponential turns out to be $$(W^Z)_{nm}=\Hom_{\sS}(F(n)\times\Delta[m]\times Z, W)$$
We define its $0$-th level to be the \textbi{mapping space} $$\Map_{\sS}(Z,W)=(W^Z)_0\cong \Hom_{\sS}(\Delta[\bullet]\times Z, W)$$ Note that this is a simplicial set.
\par We can now reapply the Yoneda lemma to this mapping space and get an isomorphism of simplicial sets $$\Map_{\sS}(F(n),W)\cong W_n$$

\section{Spaces and Kan Complexes}

\begin{defin}
    Let $n\in\bN$.
    \begin{enumerate}
        \item We define the \textbi{boundary} of $\Delta^n$ to be the simplicial set $\partial\Delta^n$ with its set of $k$-simplices being the set of non-surjective maps $[k]\to[n]$.
        \item For $0\leq i\leq n$, we define the $i$-th \textbi{horn} to be the simplicial set $\Lambda^n_i$ with its set of $k$-simplices being the set of maps $[k]\to[n]$ for which there exists a $j\ne i$ that is not contained in their image.
    \end{enumerate}
    We call the horns defined for $1<i<n-1$ the \textbi{inner horns}.
\end{defin}
In other words, the boundary is the union of all faces and the $i$-th horn is the union of all faces except the $i$-th one.

\par Since the standard $n$-simplex contains all of the maps considered above, we have inclusions $$\Lambda^n_i\hookrightarrow\partial\Delta^n\hookrightarrow\Delta^n$$

\begin{defin}
    \begin{enumerate}
        \item A map of simplicial sets $f:X\to Y$ is a (inner) \textbi{Kan fibration} if any commutative square of the form 
        \[\begin{tikzcd}
            \Lambda^n_i \arrow[r] \arrow[d, hook] & X \arrow[d, "f"] \\
            \Delta^n \arrow[ru, dashed, "\exists ?"] \arrow[r] & Y
        \end{tikzcd}\]
        admits a lift for all (inner) horn inclusions.
        \item A simplcial set $X$ is a Kan complex if it is Kan fibrant, i.e., if every horn inclusion lifts to a simplex.
        \[\begin{tikzcd}
            \Lambda^n_i \arrow[r] \arrow[d, hook] & X \\
            \Delta^n \arrow[ur, dashed]
        \end{tikzcd}\]
    \end{enumerate}   
\end{defin}

\begin{ex}
    Consider the nerve of a groupoid $N(\C)$. Here, every horn lifts to a simplex: If it is an inner horn you can use the compositions, and for the outer horns, we just use the fact that all maps are invertible.
\end{ex}

We now make the relation between Kan complexes and spaces precise. On one side, we have the \textbi{singular set} functor $\text{Sing}(-)$ from algebraic topology:

\begin{prop}
    For any topological space $T$, the singular set $\text{Sing}(T)$ is a Kan complex.
\end{prop}

On the other side, we have:

\begin{defin}
    The \textbi{geometric realization} functor $|-|:\cS\to \Top$ is the unique limit-preserving functor sending $\Delta^n$ to the topological $n$-simplex $|\Delta^n|$.
\end{defin}

\begin{rem}
    Geometric realizations are always CW-complexes.
\end{rem}

Combining the two, it can be shown that spaces and Kan complexes have the same homotopy theory:

\begin{thm}
    The singular set and geometric realization functors are adjoints, and furthermore they assemble into a Quillen equivalence.
\end{thm}

\begin{rem}
    CW-complexes are enough when working up to weak equivalence: Recall that by CW-approximation every space has the weak homotopy type of a CW-complex. Moreover, the composite $|-|\circ \text{Sing}(-)$ is a functorial CW-approximation.
\end{rem}

\section{Creating Models for $(\infty,1)$-categories}

We now want to create an analogous concept, but this time taking inspiration by our motivational example from category theory, as well as the nerve construction we outlined. We thus want a similar notion to that of a Kan complex, but where we can also work with directed maps instead of paths. 
\par However, if we take another look at our motivational observations, we may find that our possible approaches diverge in a crucial place: We haven't said if this concept of an $(\infty,1)$-category should exist as a weakened version of a Kan complex or as an object with more structure, built out of Kan complexes. For example, when thinking about Kan complexes as $\infty$-groupoids, i.e., higher versions of ordinary groupoids, then it makes sense that they should be special cases of categories. If we switch our perspective and think about spaces, then the second approach also seems reasonable. \\ 
In both cases, for a reasonable definition, we want at least the following:
\begin{itemize}
    \item A ``set'' of objects.
    \item For any two objects, a \textit{space} (meaning Kan complex) of morphisms between them.
    \item Witnesses of compositions and their properties up to homotopy.
    \item Some compatible homotopical structure: equivalences, etc.
\end{itemize}

\subsection*{The ``internal" version: Quasicategories}

\begin{defin}
    A simplicial set $X$ is a \textbi{quasicategory} if every \textit{inner} horn lifts to a simplex: 
    \[\begin{tikzcd}
        \Lambda^n_i\arrow[r] & X \\ 
        \Delta^n \arrow[from=u, hook] \arrow[ur, dashed]
    \end{tikzcd}\]
\end{defin}

\begin{rem}[Constructing quasicategories]
    So far, we haven't actually given an example of a quasicategory with nontrivial homotopical structure. To create such an object would mean to not only provide objects and (infinite levels of) morphisms, but also show how all these coherence conditions are satisfied. For that reason, we have various results and constructions which we can apply to various cases. \\ 
    Most crucially, the ``topological spaces'' example from our motivation has not been made into a quasicategory yet! For that, we use the following fact: 
    \begin{thm}[Homotopy-coherent nerve]
        There is a functor $\mathbf{N}$ taking simplicially enriched categories (i.e., categories where hom-objects are simplicial sets) to simplicial sets. If $\C$ is a simplicially enriched category such that all hom-objects are Kan complexes, then $\mathbf{N}(\C)$ is a quasicategory. 
        \\ 
        Moreover, $\mathbf{N}$ takes Dwyer-Kan equivalences to equivalences of quasicategories.
    \end{thm}
    We can then take the ordinary ($1$-)category of Kan complexes, which is Kan-complex enriched, and apply the homotopy-coherent nerve, creating an \textit{$\infty$-category of spaces}. This completes the goal of defining a category with ``spaces'', maps, homotopies, homotopies between homotopies, etc.
\end{rem}

\subsection*{The ``external" version: Complete Segal Spaces}

So, why should one not just stick to the simpler-seeming notion of a quasicateory?

\begin{itemize}
    \item[\textbf{Constructions}.] Notice that with quasicategories, we have not actually showed that basic constructions work as expected. For example, showing that the mapping spaces are actually spaces (i.e., Kan complexes) is quite involved. Providing more structure makes proving things easier. Furthermore, things get even more complicated when people need higher categories where even more levels of morphisms are not invertible.
    \item[\textbf{Type theory}.] To create a theory of $(\infty,1)$-categories in type theory, we need to define, for a given type, some type of directed arrows between terms of that type. Trying to follow the definition of a quasicategory, this would mean that we have to allow more paths to exist in identity types! Such paths would not constitute a good theory of identifications anymore, as, for example, there would not necessarily exist maps $$(x=_Ay)\to (y=_Ax)$$ witnessing the symmetry of identity. 
    \par We thus need additional structure to represent such directed paths, structure that does not come from identity types. In particular, we will introduce a \textit{directed interval} $\mathbbm{2}$ working like the 1-simplex $\Delta^1$, with maps $\mathbbm{2}\to A$ representing directed paths in $A$. 
\end{itemize}

\begin{prop}[Reedy model structure]
    There is a model structure on $\sS$ such that
    \begin{itemize}
        \item The fibrations are the maps $f:X\to Y$ for which the morphism of simplicial sets $$\Map_{\sS}(F(n),X)\to \Map_{\sS}(\partial F(n), X)\times_{\Map_{\sS}(\partial F(n),Y)}\Map_{\sS}(F(n),Y)$$ is a Kan fibration for all $n\geq 0$. 
        \item The cofibrations are the monomorphisms.
        \item The weak equivalences are the maps that are levelwise weak equivalences in $\cS_{\text{Quillen}}$.
    \end{itemize}
\end{prop}

What we are really interested in are the fibrant objects in this model structure. Unfolding the definition, we get that a simplicial space $X$ is \textbi{Reedy fibrant} if the map $$\Map_{\sS}(F(n),X)\to\Map_{\sS}(\partial F(n), X)$$ is a Kan fibration for all $n\geq 0$. 

\begin{lem}\label{refib}
    Let $X$ be a Reedy fibrant simplicial space. Then:
    \begin{enumerate}
        \item The "source-target" map $X_1\xrightarrow{(d_1,d_0)}X_0\times X_0$ is a Kan fibration.
        \item $X_n$ is a Kan complex for all $n\in \bN$.
    \end{enumerate}
\end{lem}

This definition is probably too complicated for one talk, but what we're really interested in is what happens in the lemma above, as well as a property that we mention in \ref{reedy}

\subsection*{The Segal condition}

Let $W\in\sS$. We want a coherent composition condition, so our aim is to relate the higher levels of $W$ to $W_1$. 
Notice that we have maps
$$\varphi_n : W_n\to W_1\times_{W_0}\cdots\times_{W_0}W_1$$ which we call the \textbi{Segal maps}
where $W_1\times_{W_0}\cdots\times_{W_0}W_1$ is the limit of the diagram
$$W_{\{0,1\}}\xrightarrow{d_1}W_0\xleftarrow{d_0}W_{\{1,2\}}\xrightarrow{d_1}\dots \xleftarrow{d_0}W_{\{n-2,n-1\}}\xrightarrow{d_1}W_0\xleftarrow{d_0}W_{\{n-1,n\}}$$
induced from the horizontal level of the above diagram in $\Delta$.
Here we utilized a slight abuse of notation, using $W_{\{k-1,k\}}$ to keep track of the respective map $W_n\to W_1$, induced by $[1]\xrightarrow{(k-1,k)}[n]$.

\begin{defin}
    Let $W$ be a Reedy fibrant simplicial space. $W$ is called a \textbi{Segal space} if the Segal maps 
    $$W_n\xrightarrow{\varphi_n} W_1\times_{W_0}\cdots\times_{W_0}W_1$$ are weak equivalences for all $n\geq 2$.
\end{defin}

\begin{rem}\label{reedy}
    Reedy fibrancy comes in handy again here: Using lemma \ref{refib}, it is possible to prove that the iterated pullback $W_1\times_{W_0}\cdots\times_{W_0}W_1$ is, in fact, a homotopy limit! 
\end{rem}

As before, we want our higher category to have objects, and \textit{spaces} of morphisms between them, as well as a way to reason homotopically within these structues.

\begin{defin}
    Define the set of \textbi{objects} in $W$ to be the set $Ob(W):=W_{00}$.
\end{defin}

For morphisms, note that an object, i.e., an element of $W_{00}$, is, equivalently, a map of simplicial sets $\Delta^0\to W_0$. Now, let $x,y \in Ob(W)$. We similarly represent $(x,y)$ as a map $(x,y):\Delta^0\to W_0\times W_0$. Thinking of $W_1$ as the space of all possible morphisms in $W$, we just want to pick out all the morphisms that have source $x$ and target $y$. 

\begin{defin}
    We define the \textbi{mapping space} $\Map_W(x,y)\in \cS$ to be the pullback 
\[\begin{tikzcd}
	{\Map_W(x,y)} & {W_1} \\
	{\Delta^0} & {W_0\times W_0}
	\arrow[from=1-1, to=1-2]
	\arrow["{(d_1, d_0)}", from=1-2, to=2-2]
	\arrow[from=1-1, to=2-1]
	\arrow["{(x,y)}"', from=2-1, to=2-2]
	\arrow["\lrcorner"{anchor=center, pos=0.125}, draw=none, from=1-1, to=2-2]
\end{tikzcd}\]
\end{defin}

Then, part 2 of \ref{refib} and stability of fibrations under pullback guarantees that the mapping space is actually a Kan complex. 

\begin{defin}
    The \textbi{homotopy category} $\Ho(W)$ of a Segal space $W$ is the category with 
    \begin{itemize}
        \item $Ob(\Ho(W))=Ob(W)=W_{00}$
        \item $\Hom_{\Ho(W)}(x,y)=\pi_0(\Map_W(x,y))$
    \end{itemize}
    and with composition defined by the Segal map $W_2\xrightarrow{\sim} W_1\times_{W_0}W_1$. For $x\in Ob(W)$, we define the \textbi{identity} $id_x:=[s_0(x)]$, where $s_0:W_{00}\to W_{10}$ is the degeneracy map.
    \par We say that two maps $f,g\in \Map_W(x,y)_0$ are \textbi{homotopic} ($f\sim g$) if they lie in the same path component. 
\end{defin}

Of course, we have to verify this is actually a category (it is!).

\begin{defin}
    A map $f: V \to W$ of Segal spaces is called a \textbi{Dwyer-Kan equivalence} if:
    \begin{enumerate}
        \item The induced map $\Ho(f):\Ho(V)\to \Ho(W)$ is an equivalence of categories.
        \item for any pair of objects $x,y$ in $V$, the induced map $\Map_{V}(x,y)\to \Map_{W}(fx, fy)$ is a weak equivalence of simplicial sets.
    \end{enumerate}
\end{defin}

We now reach our final problem: the Dwyer-Kan equialences we just defined are the natural form of homotopical equivalence we would expect, however Dwyer-kan equivalnces between Segal spaces are \textit{not} necessarily levelwise equivalences of Kan complexes! There is one more property we are missing, which, for those who paid attention to the homotopy type theory introduction a few weeks ago, will seem very similar to the univalence axiom for universes:

\begin{cons}
    \par Define the \textbi{space of homotopy equivalences} $W_{hoequiv}$ to be the subspace of $W_1$ consisting of all components whose points (i.e., 0-simplices) are homotopy equivalences. 
\par The degeneracy $s_0$ induces a map $s_0:W_0\to W_{hoequiv}$.
\end{cons}

\begin{defin}
    A Segal space $W$ is \textbi{complete} if $s_0:W_0\to W_{hoequiv}$ is a weak equivalence of simplicial sets.
\end{defin}

\begin{thm}
    Let $W$ be a complete Segal space and let $Ob(W)/_\sim$ denote the set of homotopy equivalence classes of objects in $\Ho(W)$. Then $\pi_0(W_0)\simeq Ob(W)/_\sim$.
\end{thm}

\begin{thm}
    Let $f:V \to W$ be a map of complete Segal spaces. Then $f$ is a Reedy equivalence (i.e., a levelwise weak equivalence) if and only if it is a Dwyer-Kan equivalence.
\end{thm}

Finally, we have a big result to bring everything together: Quasicategories and complete Segal spaces in some sense define the same homotopy theory.

\begin{thm}[Non-precise statement of (dkjshfkdjsn)]
    There are Quillen equivalences between a model structure where the fibrant objects are quasicategories and a model structure where the fibrant objects are complete Segal spaces.
\end{thm}

\end{document}